\section{State of the art}


% What's the point of new metaheuristics.
One of the {\em koans} written by Goldberg in ``Zen and the art of the
evolutionary algorithm'' states that we should Nature be our guide. This has
profitably led to many different population-based metaheuristics that are
inspired by the behavior of different species \cite{nedjah2020inspiration}, going as far as the behavior of
pigeons in a park \cite{blanco2019urban}. The exhaustion of the pool of
species with collective behavior to mimick has led further away, for instance to
zombies \cite{nguyen2012zombie}, which eventually has led to backlash
\cite{metaphor_exposed} claiming how metaphors obscure understanding of new
algorithms, and do not advance the field of optimization.

This does not imply, however, that metaphors are necessarily
unhelpful. The potential of a book such as the one we deal with here to inspire
optimization has, however, not been realized, although it has been
mentioned at least one in relation with an evolutionary algorithm:
\cite{wollam1999reverse} mentioned one of the ``methods'' of the book,
``screening out savages'', as a way of, apparently, giving 0 fitness
to missiles that didn't meet the constraints of a ``commanded flight
profile''.

% How does diversity influence search and what type of problems is going to profit from it

% What other efforts have been made in this area