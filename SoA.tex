\section{State of the art}


% What's the point of new metaheuristics.
One of the {\em koans} written by Goldberg in ``Zen and the art of the
evolutionary algorithm'' states that we should let Nature be our guide. This has
profitably led to many population-based metaheuristics that are
inspired by the behavior of different species \cite{nedjah2020inspiration}, going as far as the behavior of
pigeons in a park \cite{blanco2019urban}. The exhaustion of the pool of
species with collective behavior to mimick has led further away, for instance to
zombies \cite{nguyen2012zombie}, which eventually has led to backlash
\cite{metaphor_exposed} claiming how metaphors obscure understanding of new
algorithms, and do not advance the field of optimization. In fact, evolutionary
algorithms are not the best way to reflect social processes
\cite{chattoe1998just}, but in this case the intent of Aldous Huxley was exactly
the opposite: how an industrial, at scale, version of biological evolution
applied to the whole human race (except for what they called ``savages'')
could determine social processes.

This does not imply, however, that metaphors are necessarily
unhelpful. The potential of a book such as the one we deal with here to inspire
optimization has, however, not been realized, although it has been
mentioned at least one in relation with an evolutionary algorithm:
\cite{wollam1999reverse} mentioned one of the ``methods'' of the book,
``screening out savages'', as a way of, apparently, giving 0 fitness
to missiles that didn't meet the constraints of a ``commanded flight
profile''.

% How does diversity influence search and what type of problems is
% going to profit from it
Curiously enough, another oblique reference to Brave New World via the
sentence
\begin{quote}
  Consider the horse. They considered it.
\end{quote}
in \cite{DBLP:journals/corr/abs-2107-00314} brings us to the main
theme in this paper. By ``considering the horse'', the author of the
novel refers to how exploration, the enhancement of diversity, is able
to find new solutions to problems. Keeping diversity hight becones specially
necessary in dynamic environments \cite{cruz2011optimization} since it allows
the population to maintain a certain {\em memory} of what happened in the past.

Many different techniques have been used to enhance diversity in evolutionary
algorithms, from migration policies \cite{10.1007/978-3-642-29178-4_6} through
simple mutation \cite{jackson2011mutation} to {\em
  reactive} methods that are ``aware'' of diversity and kick-in when diversity
is low \cite{zaharie2006diversity}. However, this adds an additional layer of
complexity to the algorithm, needing the establishment of thresholds and/or
dynamic measures.

In this paper we will present a proof of concept and initial experiments for
{\sf BNA}, which preserves diversity through {\em natural} mechanisms, and uses
it to achieve higher performance in, initially, static problems.