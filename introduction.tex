\section{Introduction}

% Introduce the problem: diversity in metaheuristics/global optimization problems

At the beginning of the year one of the authors read the famous book
by Aldous Huxley: A brave new world \cite{huxley2007brave}. The novel
is a distopy that describes the development in reproductive
technology, psychological manipulation (including virtual-reality
plays called {\em feelies} \cite{lecakes2021matrix}) and classical
conditioning \cite{bernheim2002addiction}. It introduced interesting
ideas in several areas, but in this paper we are going to focus in how
they optimize the population.

In order to maximize efficiency and decrease unhappiness, the book
describes how the population is divided in \textit{castes}, assigned
since birth, where everyone knows and accepts their place. That way,
they achieve an ``optimum world``, whose optimization is based on
these reproduction restrictions and in the overall balance the division in
castes creates, not in an individual. This tension between the
individual happiness or realization and the overall harmony or optimal
state is, precisely, the main plot driver, becoming a literary
harbinger of the comparison between individual and population-based
optimization algorithms, although of course the intention of the
author was to compare collectivist and individualist culture
\cite{mathews2012happiness}.

When we talk about evolutionary algorithms the target is reaching the
optimum solution for a problem., and this books perfectly describes
the process through which they have reached the perfect human
race. Thereforth we want to develop an algorithm based on the book's
fecundation proceses and compare its behaviour with other
algorithms. Also, investigating how the division in castes affects the
poblation's diversity.

% Point at possible solutions to the problem

% Describe the layout of the rest of the paper