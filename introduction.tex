\section{Introduction}

% Introduce the problem: diversity in metaheuristics/global optimization problems
Population-based algorithms \cite{prugel2010benefits} are global optimization, stochastic
methods that use different techniques for exploring the search space
in such a way that it makes finding the solution feasible in a
reasonable amount of time. As such, they must strike a balance between
feasibility (exploitation of the features found in the current
population of solutions to optimize fitness) and off-line performance
(exploration of parts of the search space that might hold the key to
those solutions) \cite{xu2014exploration}. And in this balance,
diversity is one of the keys \cite{alba2005exploration}. This drives
the search for new algorithms that explicitly try to keep this balance
in a {\em comfort zone}.

In this search, inspiration may arrive from unexpected places.
At the beginning of the year one of the authors read the famous book
by Aldous Huxley: A brave new world \cite{huxley2007brave}. The novel
is a distopy that describes the development in reproductive
technology, psychological manipulation (including virtual-reality
plays called {\em feelies} \cite{lecakes2021matrix}) and classical
conditioning \cite{bernheim2002addiction}. It introduced interesting
ideas in several areas, but in this paper we are going to focus in how
they optimize the population.

In order to maximize efficiency and decrease unhappiness, the book
describes how the population is divided in \textit{castes}, assigned
since birth, where everyone knows and accepts their place. That way,
they achieve an ``optimum world``, whose optimization is based on
these reproduction restrictions and in the overall balance the division in
castes creates, not in an individual. This tension between the
individual happiness or realization and the overall harmony or optimal
state is, precisely, the main plot driver, becoming a literary
harbinger of the comparison between individual and population-based
optimization algorithms, although of course the intention of the
author was to compare collectivist and individualist culture
\cite{mathews2012happiness}.

When we talk about evolutionary algorithms the target is reaching the
optimum solution for a problem, and this books perfectly describes
the process through which they have reached the perfect human
race. Therefore we want to develop an algorithm based on the book's
fecundation proceses and compare its behaviour with other
algorithms. Our main assumption during the design of this process is
that the division in castes affects the
poblation's diversity; we can draw a parallel between the
collectivism/individualism tensions inherent in the novel and the
exploration/exploitation balance in population based global
algorithms, and thus try and apply their solutions to our
problems. The main conclusion we draw from the book is that what makes
collective society happy might make some individuals extremely
isolated and sad; but if we apply this to our optimization realm we
can see that what Aldous Huxley might be unwittingly proposing is the
rough layout of an optimization algorithm, where individuals with 
different fitnesses undergo different
differentiation/evolution/reproduction processes. We will take it from
here to design an optimization algorithm, which of course we call Brave
New Algorithm.

When designing an algorithm from scratch, we must also make technical
choices on how this implementation is going to take place, as well as
how the whole process of going from design (our {\em user stories}) to
final implementation (product) can be performed according to best
practices in software engineering. Implementation matters
\cite{merelo2011implementation}, and an agile development process can
help us get the final result in the most efficient way, guaranteeing
the quality of any software product \cite{DBLP:journals/corr/abs-2104-12545}.

The rest of the paper is organized as follows: next, the state of the
art in this area is examined. The algorithm (and its implementation)
are described in \ref{sec:algorithm}. The first experiments performed
with this implementation of the algorithm are shown in
\ref{sec:experiments}. Finally, we will discuss the results and
present our conclusions in Section \ref{sec:conc}.