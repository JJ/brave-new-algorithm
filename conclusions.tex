\section{Discussion and conclusions}
\label{sec:conc}

We began the paper asking ourselves how the division in castes affected
the diversity of the algorithm, with the goal in mind of making the algorithm avoid getting stuck
in local optima as late as possible. We took the entropy as the metric for measuring this, and
through the different experiments we've reached the conclusion that this division makes
the algorithm keep the diversity more generations than the basic genetic
algorithm. This leads us to our main conclusion, which is that the proof of
concept of this new algorithm contains a lot of promise for achieving good
results on a range of different optimization problems, all within a classical
evolutionary algorithm framework that uses a metaphor as inspiration, and not to
obscure the actual working of the algorithm.

Working with Julia also provided a boost to rapid development, as well as high
performance for the implementation, which was almost 3 orders of magnitude
fasters that the (admittedly naïve) implementation in Python. The genericity of
Julia will allow us to, eventually, accomodate different types of evolvable
solutions, and of course to extend it into a mainstream evolutionary algorithm,
or else make it inter-operable with one of the other Julia metaheuristic
libraries that are available right now.

As future works on this algorithm it would be interesting to label all the individuals in
the population and investigate how they move between the different castes. Also, explore
how the diversity behaves when talking about Multiobjective Optimization
Problems, as well as dynamic optimization problems.