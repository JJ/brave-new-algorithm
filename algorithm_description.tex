\section{Algorithm's Nature}
\label{sec:algorithm}

As it was mentioned before, the algorithm is based in the optimization's process of the human race described on the book, thus we
are talking about an algorithm based in the evolution of a population, it will follow the structure of evolutionary algorithms
specifically a \textit{genetic algorithm}. The book describes how they achieved the perfect human race working with an assembly 
line with different phases. This will be reflected with a \textit{generational evolutionary algorithm} with selection, 
crossover, mutation and replacement operators.

The process begins in the \textbf{\textit{Fecundation Room}}, here the eggs are created and fertilized. Once the fertilization
is finished all the eggs got to the \textbf{\textit{Hatchery}} where the caste to which each individual will belong is decided. 
Huxley describes how the higher castes (\textit{Alpha} and \textit{Beta}) are suministred a higher amount of nutrients and hormones during the 
incubation. While the lower castes (\textit{Gamma}, \textit{Delta}, and \textit{Epsilon}) are deprived of these elements, needed for the development.
To imitate this ``lack of nutrients``, in the algorithm developed we will deprive the lower castes of the operators, they will only mutate. 
With all that has been mentioned, the castes will be developed in the following way :

\begin{itemize}
    \item \textbf{Alphas}: in the books they are the most intelligents, the elite belongs to this group. They have responsibilities, they are
    ones that take decisions. In our implementation they will be reproduced with other individuals of the caste and they will evolve with
    all the operators.
    \item \textbf{Betas}: in the book they are less intelligents that the before mentioned and their main role is working in administrative tasks.
    In the implementation, the crossover will only be with individuals from the Alpha caste,
    \item \textbf{Gammas}: in the book they are subordinates, whose tasks require hability. In the implementation they will only mutate, but using local
    search
    \item \textbf{Deltas} and \textbf{Epsilons}: in the book both these castes are employees of the other castes and do repetitive works. In the 
    implementation they will only have mutation by fixed segment. 
\end{itemize}

With this structure in min the metaheuristic will be divided in the following phases :

\begin{itemize}
    \item \textbf{Fecundation room}: the individuals are created in a randomized way.
    \item \textbf{Hatchery room}: in this phase we will divide the population in castes. We will do this following the fitness value
    of the individual as the criteria. Furthermore, each caste will have a different population percentage. Because in the book they
    mention that lower castes are produced with the \textit{Bokanovsky's process}, where an ambryo its divided into 96 identical twins.
    In the algorithm this will be reflected in the population size, that will descende when the caste is higher.
    \item \textbf{Caste evolution}: each caste will follow a different process, as it was mentioned before
\end{itemize}

We are not talking about static castes, they are generated at the beginning of each generation. Let's imagine that we have a poplation size of ten, 
each individual with a fitness value. In the fist iteration the population will be divided following that value. After that, each individual will 
follow the evolution process corresponding to the caste. At the end of each generation all the chromosomes will be mixed, regardless of the caste. The
next generation will start dividing this chromosomes in castes again. 
