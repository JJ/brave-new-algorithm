\section{Implementation}

In first implementations and version of this algorithm was written in Python
\cite{merelo_molina_2021}, but a lot of performance issues arised which made the
algorithm hard to analyze. This is why for this second proof of concept, one of
the requirements was to use a language that provided more performance, and also
at the same time could have the potential for parallelism. This is why for this
version we've chosen a language that joined the ``Petaflop club``, which
includes the languages that overcome the 1 petaflop/second as peak
performance. We are talking about Julia \cite{julia} programming language, a
multiparadigm dynamically typed language. In this section we will give a bit of
insight about the data structures and implementation.

The agile manifesto has seldom been used in the scientific realm. This manifesto
has the customer at its center, and exhorts to follow best practices that
guarantee quality software through the ellaboration of minimally valuable
products; applying it to science in general as suggested by the agile science
manifesto \cite{agile_manifesto} it includes also results and any other {\em
  artifact} into the agile development cycles, with frequent interaction among
the different stakeholders.

Besides, agile science is open science, and all experiments, configuration
files, and data files, as well as obviously source, are available with an open
source license in GitHub \url{https://github.com/cecimerelo/UnAlgoritmoFeliz}.

Following best practices in this area, Domain Driven Design
\cite{evans2004domain} methodology was
applied to the problem domain. This allowed us to go from the existing user
stories (which can be read in the above mentioned GitHub repository) to the data
structures used here.

As we mentioned, Julia is a dynamically typed language, but it has some
advantages from static typed ones, making it possible to indicate the types for
some variables. So, for making a good use of this paradigm the castes have been
defined as different types, for example, for alpha caste:

\begin{lstlisting}[
    basicstyle=\small
]
    @with_kw struct ALPHA <: Caste
        name::String = "ALPHA"
    end
\end{lstlisting}

The \lstinline{with_kw} decorator is used for defining default values for the entity. Also, the castes have been defined as inmutable,
because the names will not change during the execution.

This differentiation will alow us to use the multiple dispatch, for example,
when defining the evolution operators. As we mentioned in previous sections,
depending on the castes the individual will go through different operators. In
order to avoid using conditionals to check the caste we just need to do the
following:

\begin{lstlisting}[
    basicstyle=\small
]
    selector_operator(
        caste::ALPHA, 
        caste_population
    )
    selector_operator(
        caste::BETA, 
        caste_population, 
        alpha_reproduction_pool
    )
    selector_operator(
        caste, 
        caste_population
    )
\end{lstlisting}

With the selector operator defined that way we make sure that each individual go through the corresponding process.

All the code can be found on GitHub \cite{project_repository}.


Following again best practices in the area, all functionality was individually
tested using automatic testing facilities in GitHub. Eventually, it was uploaded
to the Julia module repository, JuliaHub and is currently available to be used
from any Julia application from
\url{https://juliahub.com/ui/Packages/BraveNewAlgorithm/hZZw9/0.1.0}.
