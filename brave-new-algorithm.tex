% This is samplepaper.tex, a sample chapter demonstrating the
% LLNCS macro package for Springer Computer Science proceedings;
% Version 2.20 of 2017/10/04
%
\documentclass[runningheads]{llncs}
%
\usepackage{graphicx}
\usepackage{hyperref}
\usepackage{color}
% Used for displaying a sample figure. If possible, figure files should
% be included in EPS format.
%
% If you use the hyperref package, please uncomment the following line
% to display URLs in blue roman font according to Springer's eBook style:
\renewcommand\UrlFont{\color{blue}\rmfamily}

\begin{document}
%
\title{A brave new algorithm \thanks{This paper has been supported in part by projects DeepBio (TIN2017--85727--C4--2--P) and DemocratAI PID2020-115570GB-C22}}
%
%\titlerunning{Abbreviated paper title}
% If the paper title is too long for the running head, you can set
% an abbreviated paper title here
%
\author{First Author\inst{1}\orcidID{0000-1111-2222-3333} \and
Cecilia Merelo\inst{1}\orcidID{0000-0002-5902-0159} \and
Juan J. Merelo\inst{2}\orcidID{0000-0002-1385-9741}}
%
\authorrunning{F. Author et al.}
% First names are abbreviated in the running head.
% If there are more than two authors, 'et al.' is used.
%
\institute{
University of Granada, Granada, Spain
\email{ceciliamm@correo.ugr.es} \and
University of Granada, Granada, Spain\\
\email{jmerelo@ugr.es}}
%
\maketitle              % typeset the header of the contribution
%
\begin{abstract}

At the beginning of this year one of the authors read `A brave new
world` by Aldous Huxley.  The Nobel describes a dystopia, which
anticipates the development of breeding technology, and how this
technology creates the perfect human race. Taking into account that
when talking about genetic algorithms our goal is to achieve the
optimum solution of a problem, and this book kind of describes the
process for making the “perfect human”, or rather the ``perfect human
population'', we will try to work on this parallelism in this paper,
trying to find what is the key to the evolution processes described in
the book. The goal is to develop a Genetic algorithm based on the
fecundation process of the book and compare it to other algorithms to
see how it behaves, by investigating how the division in castes
affects the diversity in the poblation.

\keywords{Evolutionary algorithm  \and Metaheuristics \and Another keyword.}
\end{abstract}

% \input{sample_section}
\section{Introduction}

% Introduce the problem: diversity in metaheuristics/global optimization problems

At the beginning of the year one of the authors read the famous book
by Aldous Huxley: A brave new world \cite{huxley2007brave}. The novel
is a distopy that describes the development in reproductive
technology, psychological manipulation (including virtual-reality
plays called {\em feelies} \cite{lecakes2021matrix}) and classical
conditioning \cite{bernheim2002addiction}. It introduced interesting
ideas in several areas, but in this paper we are going to focus in how
they optimize the population.

In order to maximize efficiency and decrease unhappiness, the book
describes how the population is divided in \textit{castes}, assigned
since birth, where everyone knows and accepts their place. That way,
they achieve an ``optimum world``, whose optimization is based on
these reproduction restrictions and in the overall balance the division in
castes creates, not in an individual. This tension between the
individual happiness or realization and the overall harmony or optimal
state is, precisely, the main plot driver, becoming a literary
harbinger of the comparison between individual and population-based
optimization algorithms, although of course the intention of the
author was to compare collectivist and individualist culture
\cite{mathews2012happiness}.

When we talk about evolutionary algorithms the target is reaching the
optimum solution for a problem., and this books perfectly describes
the process through which they have reached the perfect human
race. Thereforth we want to develop an algorithm based on the book's
fecundation proceses and compare its behaviour with other
algorithms. Also, investigating how the division in castes affects the
poblation's diversity.

% Point at possible solutions to the problem

% Describe the layout of the rest of the paper
\section{State of the art}

% What's the point of new metaheuristics.

% How does diversity influence search and what type of problems is going to profit from it

% What other efforts have been made in this area
\section{Algorithm's Nature}
\label{sec:algorithm}

As it was mentioned before, the algorithm is based in the optimization's process of the human race described on the book, thus we
are talking about an algorithm based in the evolution of a population, it will follow the structure of evolutionary algorithms
specifically a \textit{genetic algorithm}. The book describes how they achieved the perfect human race working with an assembly 
line with different phases. This will be reflected with a \textit{generational evolutionary algorithm} with selection, 
crossover, mutation and replacement operators.

The process begins in the \textbf{\textit{Fecundation Room}}, here the eggs are created and fertilized. Once the fertilization
is finished all the eggs got to the \textbf{\textit{Hatchery}} where the caste to which each individual will belong is decided. 
Huxley describes how the higher castes (\textit{Alpha} and \textit{Beta}) are suministred a higher amount of nutrients and hormones during the 
incubation. While the lower castes (\textit{Gamma}, \textit{Delta}, and \textit{Epsilon}) are deprived of these elements, needed for the development.
To imitate this ``lack of nutrients``, in the algorithm developed we will deprive the lower castes of the operators, they will only mutate. 
With all that has been mentioned, the castes will be developed in the following way :

\begin{itemize}
    \item \textbf{Alphas}: in the books they are the most intelligents, the elite belongs to this group. They have responsibilities, they are
    ones that take decisions. In our implementation they will be reproduced with other individuals of the caste and they will evolve with
    all the operators.
    \item \textbf{Betas}: in the book they are less intelligents that the before mentioned and their main role is working in administrative tasks.
    In the implementation, the crossover will only be with individuals from the Alpha caste,
    \item \textbf{Gammas}: in the book they are subordinates, whose tasks require hability. In the implementation they will only mutate, but using local
    search
    \item \textbf{Deltas} and \textbf{Epsilons}: in the book both these castes are employees of the other castes and do repetitive works. In the 
    implementation they will only have mutation by fixed segment. 
\end{itemize}

With this structure in min the metaheuristic will be divided in the following phases :

\begin{itemize}
    \item \textbf{Fecundation room}: the individuals are created in a randomized way.
    \item \textbf{Hatchery room}: in this phase we will divide the population in castes. We will do this following the fitness value
    of the individual as the criteria. Furthermore, each caste will have a different population percentage. Because in the book they
    mention that lower castes are produced with the \textit{Bokanovsky's process}, where an ambryo its divided into 96 identical twins.
    In the algorithm this will be reflected in the population size, that will descende when the caste is higher.
    \item \textbf{Caste evolution}: each caste will follow a different process, as it was mentioned before
\end{itemize}

We are not talking about static castes, they are generated at the beginning of each generation. Let's imagine that we have a poplation size of ten, 
each individual with a fitness value. In the fist iteration the population will be divided following that value. After that, each individual will 
follow the evolution process corresponding to the caste. At the end of each generation all the chromosomes will be mixed, regardless of the caste. The
next generation will start dividing this chromosomes in castes again. 

\section{Experimental results}
\label{sec:experiments}

In this section we will run some experiments in order to see how the algorithm behaves regarding diversity.
First we will stablish which metric we will be using to measure it. Then we will analyse the brave new algorithm, and
compare it to a basic genetic algorithm with no castes division.

\subsection{Diversity analysis}

Mantaining the \emph{diversity} is crucial for avoiding the early convergence to local optima. Rosca \cite{Rosca} concluded that 
the poblations seem to get stuck in a local optima when the entropy didn't change or decreased drably in sequential generations.
In genetic programming when we talk about diversity we refer to the structural differences such as the amount of different
genotypes in the population or the singularity of the fitness values \cite{genetic}. In this section we will analyze the diversity of our
algorithm to study how division in castes affects.

There are different ways of calculating the diversity: genotypic diversity, fenotypic diversity, entropy, pseudo-isomorphism, edit distance, etc.
Among which we chose \textit{entropy}, that describes the distribution of the poblation around the different fitness values, and the \textit{edit distance},
in which each individual it's evaluated against the best individual found so far. These metrics has been chosen since according to Burke \cite{diversity} entropy
and edit distance show a great correlation with the increment and decrement of the fitness value.

\subsection{Diversity in Brave new algorithm}

For these experiments we will use the configuration shown in Table \ref{tab:config_file_10}. The fitness functions evaluated will be the Rastrigin
function from the \emph{Black-box optimization benchmarking} \cite{BBOB}.

\begin{table}[]
    \caption{Configuration parameters for the diversity exploration}
    \label{tab:config_file_10}
    \centering
    \begin{tabular}{|l|l|}
        \hline
        Configuration parameters &  Value \\
        \hline
        Chromosome size                   & 20      \\ \hline
        Population size                   & 100     \\ \hline
        Maximum generations               & 15      \\ \hline
        Alpha population percentage       & 10      \\ \hline
        Beta population percentage        & 20      \\ \hline
        Gamma population percentage       & 20      \\ \hline
        Delta population percentage       & 20      \\ \hline
        Epsilon population percentage     & 30      \\ \hline
        Mutation rate for all castes      & 10      \\ \hline
\end{tabular}
\end{table}

In Figure \ref{fig:best_restrigin_diversity} the data has been taken from the Rastrigin execution that returned the best
fitness value. In the graphic on top we have the fitness value that resulted from each generation. As we can see the 
edit distance metric is closely related to the top graphic. The smallest edit distance is in the last generation, so we
can conclude that the lower the edit distance the higher possibility of the algorithm to get stuck in a local optima. 

\begin{figure}[]
	\centering	
	\includegraphics[scale=0.5]{./figures/config_file_10_Rastrigin_diversity.png}
	\caption{ Diversity in the execution with the best fitness value for the Rastrigin function }
    \label{fig:best_restrigin_diversity}
\end{figure}

Now let's compare the entropy for the best and worst execution of the Rastrigin function to see if the entropy has 
anything to do with the algorithm outcome. As we know, algorithms seem to get stuck in local optima when
entropy doesn't change or decrease drably in sequential generations. In Figure \ref{fig:rastrigin_diversity_comparation} we can see
this case in the execution with the worst fitness value. The entropy decreases more than 2 points from generation 0 to generation 30.
Whilst in the case of the best execution the entropy slope is less strong. Also in the execution with the best fitness we can see how
when the entropy stays with similar values for multiple generations is when it gets stucked in local optima.

\begin{figure}[]
	\centering	
	\includegraphics[scale=0.5]{./figures/config_file_10_Rastrigin_diversity_comparation.png}
	\caption{ Comparation of the diversity between the execution with the best and the worst fitness value for the Rastrigin function }
    \label{fig:rastrigin_diversity_comparation}
\end{figure}

With this information we can now compare the diversity of our algorithm with the behaviour of a genetic algorithm without
castes division.

\subsection{Diversity on a basic genetic algoritm}

We will now compare the diversity of the Brave new algorithm with a ``basic`` genetic algorith, meaning an algorithm without castes division, in order
to check how the division affects the diversity.

In Table \ref{tab:diversity_comparation} we can see the results comparation for both algorithms. Having a higher value on
the entropy standard deviation means that the values are spread out over a wider range, this is the case for the Brave
new algorithm. Also, it has a better median of the fitness value. In Figure \ref{fig:rastrigin_diversity_comparation} we can see
how the entropy in the basic algorithm looks like more variant, but the values are only between 5 and 5.5.

For this section we can conclude that because the Brave new algorithm has a higher standard deviation on the entropy the 
diversity is mainted throught more generations that for the basic genetic algorithm, resulting in better median for the fitness value.

\begin{table}[]
    \centering
    \caption{Entropy results for Brave New Algorithm (BNA) and a genetic algorith without castes division (GA)}
    \begin{tabular}{|c|c|c|c|c|}
    \hline
    \textbf{Algorithm} & \textbf{\begin{tabular}[c]{@{}c@{}}entropy\\ median\end{tabular}} & \textbf{\begin{tabular}[c]{@{}c@{}}f. value\\ median\end{tabular}} & \textbf{entropy $\sigma$} & \textbf{f. value $\sigma$} \\ \hline
    BNA                & 2.9                                                               & -39.56                                                             & 0.41             & 26.12             \\ \hline
    GA                 & 5.27                                                              & 528.21                                                             & 0.01             & 224.09            \\ \hline
    \end{tabular}
    \label{tab:diversity_comparation}
\end{table}

\begin{figure}[]
	\centering	
	\includegraphics[scale=0.5]{./figures/Rastrigin_diversity_algs_comparation.png}
	\caption{ Comparation of the diversity between the best executions of the Brave New Algorithm and a genetic algorith without castes division }
    \label{fig:diversity_comparation}
\end{figure}



\section{Discussion and conclusions}
\label{sec:conc}

We began the paper asking ourselves how the division in castes affected
the diversity of the algorithm, with the goal in mind of making the algorithm avoid getting stuck
in local optima as late as possible. We took the entropy as the metric for measuring this, and
through the different experiments we've reached the conclusion that this division makes
the algorithm keep the diversity more generations than the basic genetic
algorithm. This leads us to our main conclusion, which is that the proof of
concept of this new algorithm contains a lot of promise for achieving good
results on a range of different optimization problems, all within a classical
evolutionary algorithm framework that uses a metaphor as inspiration, and not to
obscure the actual working of the algorithm.

Working with Julia also provided a boost to rapid development, as well as high
performance for the implementation, which was almost 3 orders of magnitude
fasters that the (admittedly naïve) implementation in Python. The genericity of
Julia will allow us to, eventually, accomodate different types of evolvable
solutions, and of course to extend it into a mainstream evolutionary algorithm,
or else make it inter-operable with one of the other Julia metaheuristic
libraries that are available right now.

As future works on this algorithm it would be interesting to label all the individuals in
the population and investigate how they move between the different castes. Also, explore
how the diversity behaves when talking about Multiobjective Optimization
Problems, as well as dynamic optimization problems.

\bibliographystyle{splncs04}
\bibliography{brave-new-algorithm}

\end{document}
